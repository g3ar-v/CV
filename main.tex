%============================================================================%

%
%	DOCUMENT DEFINITION
%
%============================================================================%

%we use article class because we want to fully customize the page and don't use a cv template
\documentclass[11pt,A4]{article}

%----------------------------------------------------------------------------------------
%	ENCODING
%----------------------------------------------------------------------------------------

% we use utf8 since we want to build from any machine
\usepackage[utf8]{inputenc}

%----------------------------------------------------------------------------------------
%	LOGIC
%----------------------------------------------------------------------------------------

% provides \isempty test
\usepackage{xstring, xifthen}

%----------------------------------------------------------------------------------------
%	FONT BASICS
%----------------------------------------------------------------------------------------

% some tex-live fonts - choose your own

% \usepackage[defaultsans]{droidsans}
% \usepackage[default]{comfortaa}
% \usepackage{mathptmx}
% \usepackage{newtxtext} 
\usepackage[default]{raleway}
% \usepackage{fetamont}
%\usepackage[default]{gillius}
% \usepackage[light,math]{iwona}
% \usepackage[thin]{roboto} 

% set font default
% \renewcommand*\familydefault{\sfdefault} 	
% \usepackage[T1]{fontenc}

% more font size definitions
\usepackage{moresize}

%----------------------------------------------------------------------------------------
%	FONT AWESOME ICONS
%---------------------------------------------------------------------------------------- 

% include the fontawesome icon set
\usepackage{fontawesome}
\usepackage{amssymb}

% use to vertically center content
% credits to: http://tex.stackexchange.com/questions/7219/how-to-vertically-center-two-images-next-to-each-other
\newcommand{\vcenteredinclude}[1]{\begingroup
	\setbox0=\hbox{\includegraphics{#1}}%
	\parbox{\wd0}{\box0}\endgroup}

% use to vertically center content
% credits to: http://tex.stackexchange.com/questions/7219/how-to-vertically-center-two-images-next-to-each-other
\newcommand*{\vcenteredhbox}[1]{\begingroup
	\setbox0=\hbox{#1}\parbox{\wd0}{\box0}\endgroup}

% icon shortcut
\newcommand{\icon}[3] {
	\makebox(#2, #2){\textcolor{maincol}{\csname fa#1\endcsname}}
}

% icon with text shortcut
\newcommand{\icontext}[4]{
	\vcenteredhbox{\icon{#1}{#2}{#3}}  \hspace{2pt}  \parbox{0.9\mpwidth}{\textcolor{#4}{#3}}
}

% icon with website url
\newcommand{\iconhref}[5]{
	\vcenteredhbox{\icon{#1}{#2}{#5}}  \hspace{2pt} \href{#4}{\textcolor{#5}{#3}}
}

% icon with email link
\newcommand{\iconemail}[5]{
	\vcenteredhbox{\icon{#1}{#2}{#5}}  \hspace{2pt} \href{mailto:#4}{\textcolor{#5}{#3}}
}

%----------------------------------------------------------------------------------------
%	PAGE LAYOUT  DEFINITIONS
%----------------------------------------------------------------------------------------

% page outer frames (debug-only)
% \usepackage{showframe}		

% we use paracol to display breakable two columns
\usepackage{paracol}

% define page styles using geometry
\usepackage[a4paper]{geometry}

% remove all possible margins
\geometry{top=0.2cm, bottom=0.2cm, left=0.1cm, right=0.2cm}

\usepackage{fancyhdr}
\pagestyle{empty}

% space between header and content
% \setlength{\headheight}{0pt}

% indentation is zero
\setlength{\parindent}{0mm}

%----------------------------------------------------------------------------------------
%	TABLE /ARRAY DEFINITIONS
%---------------------------------------------------------------------------------------- 

% extended aligning of tabular cells
\usepackage{array}

% custom column right-align with fixed width
% use like p{size} but via x{size}
\newcolumntype{x}[1]{%
>{\raggedleft\hspace{0pt}}p{#1}}%


%----------------------------------------------------------------------------------------
%	GRAPHICS DEFINITIONS
%---------------------------------------------------------------------------------------- 

%for header image
\usepackage{graphicx}

% use this for floating figures
% \usepackage{wrapfig}
% \usepackage{float}
% \floatstyle{boxed} 
% \restylefloat{figure}

%for drawing graphics		
\usepackage{tikz}
\usetikzlibrary{shapes, backgrounds,mindmap, trees}

%----------------------------------------------------------------------------------------
%	Color DEFINITIONS
%---------------------------------------------------------------------------------------- 
\usepackage{transparent}
\usepackage{color}

% primary color
\definecolor{maincol}{RGB}{ 55, 57, 63 }

% accent color, secondary
% \definecolor{accentcol}{RGB}{ 250, 150, 10 }

% dark color
\definecolor{darkcol}{RGB}{ 70, 70, 70 }

% page background color
\definecolor{pagecol}{RGB}{ 212, 202, 177}

% light color
\definecolor{lightcol}{RGB}{170,162,142}


% Package for links, must be the last package used
\usepackage[hidelinks]{hyperref}

% returns minipage width minus two times \fboxsep
% to keep padding included in width calculations
% can also be used for other boxes / environments
\newcommand{\mpwidth}{\linewidth-\fboxsep-\fboxsep}



%============================================================================%
%
%	CV COMMANDS
%
%============================================================================%

%----------------------------------------------------------------------------------------
%	 CV LIST
%----------------------------------------------------------------------------------------

% renders a standard latex list but abstracts away the environment definition (begin/end)
\newcommand{\cvlist}[1] {
	\begin{itemize}\item {#1}\end{itemize}
}

%----------------------------------------------------------------------------------------
%	 CV TEXT
%----------------------------------------------------------------------------------------

% base class to wrap any text based stuff here. Renders like a paragraph.
% Allows complex commands to be passed, too.
% param 1: *any
\newcommand{\cvtext}[1] {
	\begin{tabular*}{1\mpwidth}{p{0.98\mpwidth}}
		\parbox{1\mpwidth}{#1}
	\end{tabular*}
}

%----------------------------------------------------------------------------------------
%	CV SECTION
%----------------------------------------------------------------------------------------

% Renders a CV section headline with a nice underline in main color.
% param 1: section title
\newcommand{\cvsection}[1] {
	\cvtext{
		\textbf{\LARGE{\textcolor{darkcol}{\uppercase{#1}}}}\\[-6pt]
		\textcolor{maincol}{ \rule{0.1\textwidth}{2pt} } \\
	}

}
%----------------------------------------------------------------------------------------
%	CV SEMISECTION
%----------------------------------------------------------------------------------------

% Renders a CV section headline with a nice underline in main color.
% param 1: section title
\newcommand{\cvsubsection}[1] {
	\vspace{9pt}
	\cvtext{
		\textbf{\normalsize{\textcolor{darkcol}{\uppercase{#1}}}}\\[-4pt]
		\textcolor{maincol}{ \rule{0.1\textwidth}{2pt} } \\
	}
}

%----------------------------------------------------------------------------------------
%	META SKILL
%----------------------------------------------------------------------------------------

% Renders a progress-bar to indicate a certain skill in percent.
% param 1: name of the skill / tech / etc.
% param 2: level (for example in years)
% param 3: percent, values range from 0 to 1
\newcommand{\cvskill}[2] {
	\begin{tabular*}{1\mpwidth}{p{0.72\mpwidth}  r}
		\textcolor{darkcol}{\textbf{#1}} & \textcolor{maincol}{#2}
	\end{tabular*}

	% \hspace{4pt}
	% \begin{tikzpicture}[scale=1,rounded corners=2pt,damn thin]
	% 	% \fill [lightcol] (0,0) rectangle (1\mpwidth, 0.15);
	% 	% \fill [maincol] (0,0) rectangle (#3\mpwidth, 0.15);
	% \end{tikzpicture}%
}


%----------------------------------------------------------------------------------------
%	 CV EVENT
%----------------------------------------------------------------------------------------

% Renders a table and a paragraph (cvtext) wrapped in a parbox (to ensure minimum content
% is glued together when a pagebreak appears).
% Additional Information can be passed in text or list form (or other environments).
% the work you did
% param 1: time-frame i.e. Sep 14 - Jan 15 etc.
% param 2:	 event name (job position etc.)
% param 3: Customer, Employer, Industry
% param 4: Short description
% param 5: work done (optional)
% param 6: technologies include (optional)
% param 7: achievements (optional)
\newcommand{\cvevent}[7] {

	% we wrap this part in a parbox, so title and description are not separated on a pagebreak
	% if you need more control on page breaks, remove the parbox
	\parbox{\mpwidth}{
		\begin{tabular*}{1\mpwidth}{p{0.72\mpwidth}  r}
			\textcolor{black}{\textbf{#2}} & \colorbox{darkcol}{\makebox[0.26\mpwidth]{\textcolor{pagecol}{#1}}} \\
			\textcolor{maincol}{\textbf{#3}} & \\
		\end{tabular*}\\[1pt]
		\ifthenelse{\isempty{#4}}{}{
			\cvtext{#4}\\
		}

		\ifthenelse{\isempty{#5}}{}{
			\cvtext{#5}
		}

		\ifthenelse{\isempty{#6}}{}{
			\cvtext{#6}\\
		}

	}
}

%----------------------------------------------------------------------------------------
%	 CV META EVENT
%--------------------------------------------------------------------------------------o

% Renders a CV event on the sidebar
% param 1: title
% param 2: subtitle (optional)
% param 3: customer, employer, etc,. (optional)
% param 4: info text (optional)
\newcommand{\cvmetaevent}[4] {
	\textcolor{maincol} {\cvtext{\textbf{\begin{flushleft}#1\end{flushleft}}}}

	\ifthenelse{\isempty{#2}}{}{
		\textcolor{darkcol} {\cvtext{\textbf{#2}} }
	}

	\ifthenelse{\isempty{#3}}{}{
		\cvtext{{ \textcolor{darkcol} {#3} }}\\
	}

	\cvtext{#4}\\[14pt]
}

%---------------------------------------------------------------------------------------
%	QR CODE
%----------------------------------------------------------------------------------------

% Renders a qrcode image (centered, relative to the parentwidth)
% param 1: percent width, from 0 to 1
\newcommand{\cvqrcode}[1] {
	\begin{center}
		\includegraphics[width={#1}\mpwidth]{qrcode}
	\end{center}
}

%============================================================================%
%
%	DOCUMENT CONTENT
%
%============================================================================%
\begin{document}
\columnratio{0.36}
\setlength{\columnsep}{2.2em}
\setlength{\columnseprule}{4pt}
\colseprulecolor{lightcol}
\pagecolor{pagecol}
\begin{paracol}{2}
	\begin{leftcolumn}
		%---------------------------------------------------------------------------------------
		%	META IMAGE
		%----------------------------------------------------------------------------------------
		%\includegraphics[width=\linewidth]{untitled.jpg}	%trimming relative to image size

		\vfill\null
		\cvsection{CONTACT}

		\iconemail{EnvelopeSquare}{14}{vfranktor@gmail.com}{vfranktor@gmail.com}{black}\\[6pt]
		\icontext{Github}{14}{\href{https://github.com/g3ar-v}{g3ar-v}}{black}\\[6pt]
		\icontext{Linkedin}{14}{\href{https://www.linkedin.com/in/victor-nyoyoko-1a1518196}{victor nyoyoko}}{black}\\[6pt]
		\icontext{Phone}{14}{+44 7760532407}{black}\\[6pt]
		\icontext{Link}{14}{\href{https://g3ar.vercel.app}{https://g3ar.vercel.app}}{black}\\[6pt]
		% \vfill\null
		%\cvqrcode{0.7}

		%---------------------------------------------------------------------------------------
		%	META SKILLS
		%----------------------------------------------------------------------------------------
		\cvsubsection{AREA OF EXPERTISE}

		\cvskill{Machine Learning}{3 years}
		\cvskill{Speech Recognition}{2 year}
		\cvskill{Automation}{1 year}
		\cvskill{Web Development}{1 year}
		\cvskill{Computer Vision}{1 year}
		\cvskill{Backend Development}{2 year}
		\cvskill{Embedded-programming}{2 years}

		% \vfill\null
		\cvsubsection{FRAMEWORKS \& TOOLS}

		\cvtext{\textbf{ML tools}: jupyter-notebook, keras, numpy, pandas, pytorch, tensorflow}\\
		\cvtext{\textbf{Linux scripting}: zsh, bash}\\
		\cvtext{\textbf{API testing}: Postman}\\
		\cvtext{\textbf{Database}:Azure cloud services, MySQL, T-SQL}\\
		\cvtext{\textbf{Frameworks \& Libraries}: Django, Flask, ReactJS, NodeJS, NextJS}\\
		\cvtext{\textbf{Editor}: NeoVim, VsCode, Matlab}\\

		%---------------------------------------------------------------------------------------
		%	SKILLS
		%----------------------------------------------------------------------------------------
		\cvsubsection{LANGUAGES}

		\cvskill{C}{5 years}
		\cvskill{C++}{3 years}
		\cvskill{R}{1 year}
		\cvskill{Lua}{1 year}
		\cvskill{Java}{5 years}
		\cvskill{Latex}{3 years}
		\cvskill{Python}{5 years}
		\cvskill{JavaScript}{3 years}

		%\cvqrcode{0.7}

		%---------------------------------------------------------------------------------------
		%	CERTIFICATIONS
		%----------------------------------------------------------------------------------------
		\cvsubsection{CERTIFICATIONS}
		\vspace{-17pt}
		\cvmetaevent
		{Professional Machine Learning Engineer}
		{Google Cloud}
		{}
		{- in progress -}
		\newpage
		\vfill\null
		%---------------------------------------------------------------------------------------
		%	ACHIEVEMENTS
		%----------------------------------------------------------------------------------------
		\cvsubsection{ACHIEVEMENTS}
		\vspace{-17pt}
		\cvmetaevent
		{University of Birmingham}
		{International Achievement Bursary and Excellence Scholarship}
		{}
		{Excellent academic record while gaining admission into the University of Birmingham}
		\nopagebreak
	\end{leftcolumn}
	\begin{rightcolumn}
		%---------------------------------------------------------------------------------------
		%	TITLE  HEADER
		%----------------------------------------------------------------------------------------
		\fcolorbox{pagecol}{darkcol}{\begin{minipage}[c][2.0cm][c]{1\mpwidth}
				\begin {center}
				\HUGE{ \textbf{ \textcolor{pagecol}{ \uppercase{ VICTOR NYOYOKO } } } } \\[-28pt]
				\textcolor{pagecol}{ \rule{0.1\textwidth}{1.25pt} } \\[4pt]
				\large{ \textcolor{pagecol} {Software/Machine Learning Engineer} }
				\end {center}
			\end{minipage}} \\[14pt]
		% \vspace{-10pt}
		%---------------------------------------------------------------------------------------
		%	PERSONAL STATEMENT
		%----------------------------------------------------------------------------------------

		\cvsection{PERSONAL STATEMENT}
		\cvtext
		{I am a 22 years old software engineer, driven by an unceasing curiousity for engineering: the need to know how anything functions, how it is implemented and to contribute to its further development. This has steered my learning in the art and science of applying programming paradigms, social skills and analytical thinking. A position in your organisation will enable me, impart my skills and values while expanding my knowledge and experience }
		% \vspace{13pt}
		\vfill\null
		%---------------------------------------------------------------------------------------
		%	PROJECTS
		% what did itsolve
		% what does it do
		% how does it impact the world
		%----------------------------------------------------------------------------------------



		\cvsection{PROJECTS}
		\cvevent
		{\textbf{2022 - Currently}}
		{T.R.E.V.O.R (virtual assistant)}
		{Tools: Raspberrypi, MycroftAI, Python, Bash, Linux, Deep Learning, API integration}
		{$\nrightarrow$ A modular AI assistant designed with the foundation of MycroftAI that automates, manages productivity and provides information}
		{$\nrightarrow$ A recent integration with gpt-3.5 provides information as quick as possible. }
		{$\nrightarrow$ The recent focus is automating system processes with NLP (Natural
			Language Processing) and instruction fine-tuning: which mimics the ability for it to
			think for itself while running processes}


		\cvevent
		{\textbf{2022 - Currently}}
		{Webporftolio/blog}
		{Tools: React, NextJS, chakraUI, Vercel}
		{$\nrightarrow$ A web portfolio/blog to showcase my projects and skills}
		{$\nrightarrow$ The aim is to create a platform to share my knowledge and experience with the world}
		{$\nrightarrow$ Recent focus is the design and implementation of the blog section}


		\cvevent
		{\textbf{2022}}
		{Attendance monitoring system}
		{Tools: Python, Raspberry Pi, ReactJS, NodeJS, Azure, JavaScript, SQLserver, Flask}
		{$\nrightarrow$ Created a novel solution that solves the unreliablity of internet connectivity in attendance monitoring systems.}
		{$\nrightarrow$ Uses RFID and fingerprint recognition to identify students}
		{$\nrightarrow$ Uses Azure services to store data and gives the admin access to track attendance over the Internet}


		\cvevent
		{\textbf{2018}}
		{Super flyer}
		{Tool: Game Development, Java, JavaFx, Networking}
		{$\nrightarrow$ A team project to create a multiplayer game with JavaFx and networking}
		{$\nrightarrow$ A relative gameplay compared to mario and donkey kong but with a vertically oriented smooth scroll layer }
		{$\nrightarrow$ Implemented the networking component of the game with TCP sockets, prioritizing speed against data transmission}
		\vfill\null
		\vfill\null
		\vfill\null
		\vfill\null

		%---------------------------------------------------------------------------------------
		%	WORK EXPERIENCE
		%----------------------------------------------------------------------------------------
		\newpage
		\vfill\null
		\cvsection{EXPERIENCE}
		\cvevent
		{\textbf{Mar 20 - Aug 21}}
		{Freelancer}
		{Fiverr}
		{A range of software development. Bug hunting in small to large \\codebases,
			whitebox testing, developing projects for clients mainly in Java, C and Python}
		\vfill\null

		%---------------------------------------------------------------------------------------
		%	EDUCATION
		%----------------------------------------------------------------------------------------
		\vfill\null
		\cvsection{EDUCATION}
		\cvevent
		{\textbf{2018 - 2022}}
		{MEng. Computer Science $\&$ Software Engineering}
		{University of Birmingham - Birmingham, United Kingdom}
		{\textbf{Some modules studied}: Data Structures and Algorithm, Machine Learning, Networking, Natural Language processing, Artificial Intelligence, Computer Vision, Security of Real-World Systems}
		{Graduated with a Second Class Upper Division (2:1)}
		\vfill\null

	\end{rightcolumn}
\end{paracol}
\end{document}

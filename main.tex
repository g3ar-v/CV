%============================================================================%
%
%	DOCUMENT DEFINITION
%
%============================================================================%

%we use article class because we want to fully customize the page and don't use a cv template
\documentclass[9pt,A4]{article}	


%----------------------------------------------------------------------------------------
%	ENCODING
%----------------------------------------------------------------------------------------

% we use utf8 since we want to build from any machine
\usepackage[utf8]{inputenc}		

%----------------------------------------------------------------------------------------
%	LOGIC
%----------------------------------------------------------------------------------------

% provides \isempty test
\usepackage{xstring, xifthen}

%----------------------------------------------------------------------------------------
%	FONT BASICS
%----------------------------------------------------------------------------------------

% some tex-live fonts - choose your own

%\usepackage[defaultsans]{droidsans}
%\usepackage[default]{comfortaa}
%\usepackage{cmbright}
\usepackage[default]{raleway}
%\usepackage{fetamont}
%\usepackage[default]{gillius}
%\usepackage[light,math]{iwona}
%\usepackage[thin]{roboto} 

% set font default
\renewcommand*\familydefault{\sfdefault} 	
\usepackage[T1]{fontenc}

% more font size definitions
\usepackage{moresize}

%----------------------------------------------------------------------------------------
%	FONT AWESOME ICONS
%---------------------------------------------------------------------------------------- 

% include the fontawesome icon set
\usepackage{fontawesome}

% use to vertically center content
% credits to: http://tex.stackexchange.com/questions/7219/how-to-vertically-center-two-images-next-to-each-other
\newcommand{\vcenteredinclude}[1]{\begingroup
\setbox0=\hbox{\includegraphics{#1}}%
\parbox{\wd0}{\box0}\endgroup}

% use to vertically center content
% credits to: http://tex.stackexchange.com/questions/7219/how-to-vertically-center-two-images-next-to-each-other
\newcommand*{\vcenteredhbox}[1]{\begingroup
\setbox0=\hbox{#1}\parbox{\wd0}{\box0}\endgroup}

% icon shortcut
\newcommand{\icon}[3] { 							
	\makebox(#2, #2){\textcolor{maincol}{\csname fa#1\endcsname}}
}	

% icon with text shortcut
\newcommand{\icontext}[4]{ 						
	\vcenteredhbox{\icon{#1}{#2}{#3}}  \hspace{2pt}  \parbox{0.9\mpwidth}{\textcolor{#4}{#3}}
}

% icon with website url
\newcommand{\iconhref}[5]{ 						
    \vcenteredhbox{\icon{#1}{#2}{#5}}  \hspace{2pt} \href{#4}{\textcolor{#5}{#3}}
}

% icon with email link
\newcommand{\iconemail}[5]{ 						
    \vcenteredhbox{\icon{#1}{#2}{#5}}  \hspace{2pt} \href{mailto:#4}{\textcolor{#5}{#3}}
}

%----------------------------------------------------------------------------------------
%	PAGE LAYOUT  DEFINITIONS
%----------------------------------------------------------------------------------------

% page outer frames (debug-only)
% \usepackage{showframe}		

% we use paracol to display breakable two columns
\usepackage{paracol}

% define page styles using geometry
\usepackage[a4paper]{geometry}

% remove all possible margins
\geometry{top=1cm, bottom=1cm, left=1cm, right=1cm}

\usepackage{fancyhdr}
\pagestyle{empty}

% space between header and content
% \setlength{\headheight}{0pt}

% indentation is zero
\setlength{\parindent}{0mm}

%----------------------------------------------------------------------------------------
%	TABLE /ARRAY DEFINITIONS
%---------------------------------------------------------------------------------------- 

% extended aligning of tabular cells
\usepackage{array}

% custom column right-align with fixed width
% use like p{size} but via x{size}
\newcolumntype{x}[1]{%
>{\raggedleft\hspace{0pt}}p{#1}}%


%----------------------------------------------------------------------------------------
%	GRAPHICS DEFINITIONS
%---------------------------------------------------------------------------------------- 

%for header image
\usepackage{graphicx}

% use this for floating figures
% \usepackage{wrapfig}
% \usepackage{float}
% \floatstyle{boxed} 
% \restylefloat{figure}

%for drawing graphics		
\usepackage{tikz}				
\usetikzlibrary{shapes, backgrounds,mindmap, trees}

%----------------------------------------------------------------------------------------
%	Color DEFINITIONS
%---------------------------------------------------------------------------------------- 
\usepackage{transparent}
\usepackage{color}

% primary color
\definecolor{maincol}{RGB}{ 45, 50, 90 }

% accent color, secondary
% \definecolor{accentcol}{RGB}{ 250, 150, 10 }

% dark color
\definecolor{darkcol}{RGB}{ 70, 70, 70 }

% light color
\definecolor{lightcol}{RGB}{245,245,245}


% Package for links, must be the last package used
\usepackage[hidelinks]{hyperref}

% returns minipage width minus two times \fboxsep
% to keep padding included in width calculations
% can also be used for other boxes / environments
\newcommand{\mpwidth}{\linewidth-\fboxsep-\fboxsep}
	


%============================================================================%
%
%	CV COMMANDS
%
%============================================================================%

%----------------------------------------------------------------------------------------
%	 CV LIST
%----------------------------------------------------------------------------------------

% renders a standard latex list but abstracts away the environment definition (begin/end)
\newcommand{\cvlist}[1] {
	\begin{itemize}{#1}\end{itemize}
}

%----------------------------------------------------------------------------------------
%	 CV TEXT
%----------------------------------------------------------------------------------------

% base class to wrap any text based stuff here. Renders like a paragraph.
% Allows complex commands to be passed, too.
% param 1: *any
\newcommand{\cvtext}[1] {
	\begin{tabular*}{1\mpwidth}{p{0.98\mpwidth}}
		\parbox{1\mpwidth}{#1}
	\end{tabular*}
}

%----------------------------------------------------------------------------------------
%	CV SECTION
%----------------------------------------------------------------------------------------

% Renders a a CV section headline with a nice underline in main color.
% param 1: section title
\newcommand{\cvsection}[1] {
	\vspace{12pt}
	\cvtext{
		\textbf{\LARGE{\textcolor{darkcol}{\uppercase{#1}}}}\\[-4pt]
		\textcolor{maincol}{ \rule{0.1\textwidth}{2pt} } \\
	}
}
%----------------------------------------------------------------------------------------
%	CV SEMISECTION
%----------------------------------------------------------------------------------------

% Renders a a CV section headline with a nice underline in main color.
% param 1: section title
\newcommand{\cvsubsection}[1] {
	\vspace{9pt}
	\cvtext{
		\textbf{\normalsize{\textcolor{darkcol}{\uppercase{#1}}}}\\[-4pt]
		\textcolor{maincol}{ \rule{0.1\textwidth}{2pt} } \\
	}
}

%----------------------------------------------------------------------------------------
%	META SKILL
%----------------------------------------------------------------------------------------

% Renders a progress-bar to indicate a certain skill in percent.
% param 1: name of the skill / tech / etc.
% param 2: level (for example in years)
% param 3: percent, values range from 0 to 1
\newcommand{\cvskill}[3] {
	\begin{tabular*}{1\mpwidth}{p{0.72\mpwidth}  r}
 		\textcolor{black}{\textbf{#1}} & \textcolor{maincol}{#2}\\
	\end{tabular*}%
	
	\hspace{4pt}
	\begin{tikzpicture}[scale=1,rounded corners=2pt,very thin]
		\fill [lightcol] (0,0) rectangle (1\mpwidth, 0.15);
		\fill [maincol] (0,0) rectangle (#3\mpwidth, 0.15);
  	\end{tikzpicture}%
}


%----------------------------------------------------------------------------------------
%	 CV EVENT
%----------------------------------------------------------------------------------------

% Renders a table and a paragraph (cvtext) wrapped in a parbox (to ensure minimum content
% is glued together when a pagebreak appears).
% Additional Information can be passed in text or list form (or other environments).
% the work you did
% param 1: time-frame i.e. Sep 14 - Jan 15 etc.
% param 2:	 event name (job position etc.)
% param 3: Customer, Employer, Industry
% param 4: Short description
% param 5: work done (optional)
% param 6: technologies include (optional)
% param 7: achievements (optional)
\newcommand{\cvevent}[7] {
	
	% we wrap this part in a parbox, so title and description are not separated on a pagebreak
	% if you need more control on page breaks, remove the parbox
	\parbox{\mpwidth}{
		\begin{tabular*}{1\mpwidth}{p{0.72\mpwidth}  r}
	 		\textcolor{black}{\textbf{#2}} & \colorbox{maincol}{\makebox[0.25\mpwidth]{\textcolor{white}{#1}}} \\
			\textcolor{maincol}{\textbf{#3}} & \\
		\end{tabular*}\\[4pt]
	
		\ifthenelse{\isempty{#4}}{}{
			\cvtext{#4}\\
		}
	}

	\ifthenelse{\isempty{#5}}{}{
		\vspace{4pt}
		{#5}
	}
	\vspace{4pt}
}

%----------------------------------------------------------------------------------------
%	 CV META EVENT
%----------------------------------------------------------------------------------------

% Renders a CV event on the sidebar
% param 1: title
% param 2: subtitle (optional)
% param 3: customer, employer, etc,. (optional)
% param 4: info text (optional)
\newcommand{\cvmetaevent}[4] {
	\textcolor{maincol} {\cvtext{\textbf{\begin{flushleft}#1\end{flushleft}}}}

	\ifthenelse{\isempty{#2}}{}{
	\textcolor{darkcol} {\cvtext{\textbf{#2}} }
	}

	\ifthenelse{\isempty{#3}}{}{
		\cvtext{{ \textcolor{darkcol} {#3} }}\\
	}

	\cvtext{#4}\\[14pt]
}

%---------------------------------------------------------------------------------------
%	QR CODE
%----------------------------------------------------------------------------------------

% Renders a qrcode image (centered, relative to the parentwidth)
% param 1: percent width, from 0 to 1
\newcommand{\cvqrcode}[1] {
	\begin{center}
		\includegraphics[width={#1}\mpwidth]{qrcode}
	\end{center}
}

%=+=+=+=+=+=+=+=+=+=+=+=+=+=+=+=+=+=+=+=+=+=+=+=+=+=+=+=+=+=+=+=+=+=+=+=+=+=+=+=+
%,,,,,,,,,,,,,,,,,,,,,,,,,,,,,,,,,,,,,,,,,,,,,,,,,,,,,,,,,,,,,,,,,,,,,,,,,,,,,,,,
                       % EDIT AFTER THIS POINT
%''''''''''''''''''''''''''''''''''''''''''''''''''''''''''''''''''''''''''''''''
%=+=+=+=+=+=+=+=+=+=+=+=+=+=+=+=+=+=+=+=+=+=+=+=+=+=+=+=+=+=+=+=+=+=+=+=+=+=+=+=+


%============================================================================%
%
%
%
%	DOCUMENT CONTENT
%
%
%
%============================================================================%
\begin{document}
\columnratio{0.31}
\setlength{\columnsep}{2.2em}
\setlength{\columnseprule}{4pt}
\colseprulecolor{lightcol}
\begin{paracol}{2}
\begin{leftcolumn}
%---------------------------------------------------------------------------------------
%	META IMAGE
%----------------------------------------------------------------------------------------
%\includegraphics[width=\linewidth]{untitled.jpg}	%trimming relative to image size


\vfill\null
\cvsection{CONTACT}
	
\iconemail{EnvelopeSquare}{14}{vfranktor@gmail.com}{vfranktor@gmail.com}{black}\\[6pt]
\icontext{Github}{14}{\href{https://github.com/g3ar-v}{g3ar-v}}{black}\\[6pt]
\icontext{Linkedin}{14}{\href{https://www.linkedin.com/in/victor-nyoyoko-1a1518196}{victor nyoyoko}}{black}\\[6pt]
\icontext{Phone}{14}{+44 7760532407}{black}\\[6pt]
% \vfill\null
%\cvqrcode{0.7}

%---------------------------------------------------------------------------------------
%	META SKILLS
%----------------------------------------------------------------------------------------
\cvsubsection{AREA OF EXPERTISE}

\cvtext{Internet of Things/Automation}

\cvtext{Computer Vision}

\cvtext{Back-end development}

\cvtext{FullStack development}

\cvtext{Front-end development}

\cvtext{Cloud Computing}

\cvtext{Embedded Programming}

% \vfill\null
\cvsubsection{FRAMEWORKS \& TOOLS}

%\cvskill{Skill_Name} {Years of experience} {percentage of bar fill} \\[-2pt]
\cvtext{\textbf{Version Control}: Git, Gitlab, Github}

\cvtext{\textbf{Linux scripting}: zsh, bash, fish}

\cvtext{\textbf{API testing}: Postman}

\cvtext{\textbf{Database}: Azure (SQLserver), MySQL, T-SQL}

\cvtext{\textbf{Frameworks}: Django, Flask, ReactJS, NodeJS, NextJS}




%---------------------------------------------------------------------------------------
%	SKILLS
%----------------------------------------------------------------------------------------
\cvsubsection{LANGUAGES}

%\cvskill{Skill_Name} {Years of experience} {percentage of bar fill} \\[-2pt]

\cvtext{C}

\cvtext{C++}

\cvtext{Lua}

\cvtext{Java}

\cvtext{Latex}

\cvtext{Python}

\cvtext{Javascript}

%\cvqrcode{0.7}

%---------------------------------------------------------------------------------------
%	ACHIEVEMENTS
%----------------------------------------------------------------------------------------
\newpage
\vfill
\cvsubsection{ACHIEVEMENTS}

\cvmetaevent
{University of Birmingham}
{International Achievement Bursary and Excellence Scholarship}
{}
{Excellent academic record while gaining admission into the University of Birmingham}

\end{leftcolumn}
\begin{rightcolumn}
%---------------------------------------------------------------------------------------
%	TITLE  HEADER
%----------------------------------------------------------------------------------------
\fcolorbox{white}{darkcol}{\begin{minipage}[c][2.5cm][c]{1\mpwidth}
	\begin {center}
		\HUGE{ \textbf{ \textcolor{white}{ \uppercase{ VICTOR NYOYOKO } } } } \\[-24pt]
		\textcolor{white}{ \rule{0.1\textwidth}{1.25pt} } \\[4pt]
		\large{ \textcolor{white} {Software Engineer} }
	\end {center}
\end{minipage}} \\[14pt]
\vspace{-12pt}
%---------------------------------------------------------------------------------------
%	PERSONAL STATEMENT
%----------------------------------------------------------------------------------------
\vfill\null
\cvsection{PERSONAL STATEMENT} 

\cvtext
{I am driven by an unceasing curiousity for engineering: the need to know how anything function, how it is implemented and to contribute to its further development. This has steered my learning in the art of analytical thinking, applying programming paradigms and social skills. A position in your organisation will enable me, impart my skills and values while expanding my knowledge and experience }
\vfill\null
%---------------------------------------------------------------------------------------
%	WORK EXPERIENCE
%----------------------------------------------------------------------------------------

\cvsection{EXPERIENCE}
\cvevent
	{\textbf{Mar 21 - Aug 21}}
	{Freelancer}
	{Fiverr}
	{A range of development. From fixing minor bugs, whitebox testing to creating projects for clients}
	\vfill\null

	\begin{itemize}
		\item \url{https://www.fiverr.com/victornyoyoko?up_rollout=true}
	\end{itemize}

%---------------------------------------------------------------------------------------
%	EDUCATION
%----------------------------------------------------------------------------------------

%\vfill\null
\cvsection{EDUCATION}

\cvevent
	{\textbf{2018 - 2022}}
	{MEng. Computer Science $\&$ Engineering}
	{University of Birmingham - Birmingham, United Kingdom}
	{\textbf{Some modules studied}: Data Structures and Algorithm, Machine Learning, Networking, Mobile Computing, Logic and Computation, Functional Programming, Computer Vision, Security of Real-World Systems}
\vfill\null

\cvtext{Status: Completed dissertation and Final exam (2:1)}

%---------------------------------------------------------------------------------------
%	PROJECTS
%----------------------------------------------------------------------------------------
% \vfill\null
\cvsection{PROJECTS}
%%what did itsolve
%%what does it do
%% how does it impact the world
\cvevent
	{\textbf{Currently}}
	{Website Portfolio}
	{Tool: ReactJS, NextJS, chakra-UI, Jest(unit-testing) }
	{With the aim to learn server-side rendering with NextJS and provide details about myself and my project }
\vfill\null


\cvevent
	{\textbf{2022}}
	{Attendance monitoring system}
	{Tool: Python, Raspberry Pi, ReactJS, NodeJS, Azure, JavaScript, SQLserver}
	{Created a novel solution that solves the unreliablity of internet connectivity in attendance monitoring systems.}
\vfill\null

\begin{itemize}
	\item Wrote a research paper on the design process and the implementation of this work
	\item An agile development cycle was used for the implementation of this project
	\item Gained an in-depth knowledge and skill in multiple fields that were require for the project
 	\item \url{https://github.com/g3ar-v/UOBproject.git}
\end{itemize}

\cvevent
	{\textbf{2018}}
	{Super flyer}
	{Tool: Game Development, Java, JavaFx, Networking}
	{A team project that required creating a game with JavaFx}
\vfill\null

 \begin{itemize}
	\item The game was of a mulitplayer game with a similar style to donkey kong and super mario combined.
	\item Implemented the networking component of the game, this required an understandng of how all components worked collectively(Rendering, UI)
	\item Reinforced my knowledge of networking and the applications of object-oriented programming in game design.
\end{itemize}

% hotfixes to create fake-space to ensure the whole height is used
\vfill
\vfill
\vfill
\end{rightcolumn}
\end{paracol}
\end{document}

